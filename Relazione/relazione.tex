\documentclass[10pt,a4paper]{article}

\usepackage[english,italian]{babel}
\usepackage[utf8]{inputenc}
\usepackage[T1]{fontenc}

\usepackage{comment}

%------Liste------
\usepackage{enumitem} %lists

%------Figure------
\usepackage{graphicx} %img/media
\usepackage{wrapfig} %per wrappare il testo attorno alle immagini

%------Math------
\usepackage{mathtools} %math package
\usepackage{amssymb} %pacchetto per i simboli di matematica
\usepackage{amsmath} %pacchetto di matematica
\usepackage{amsthm} %pacchetto di matematica\raggedbottom

%------CodeEnviroment----
\usepackage{listings} %coding environment 
\usepackage{xcolor} %define colors

%-------Margini------
\usepackage{geometry}
%\geometry{a4paper,top=1.5cm,bottom=2.5cm,left=1.5cm,right=1.5cm,heightrounded}
\raggedbottom

%------Tabelle------
\usepackage{booktabs} %toprule e midrule usati al posto di hline
\usepackage{float} %oggetti mobili(tabelle, figure etc)
\usepackage{longtable,tabu} %tabelle lunghe su più pagine
\usepackage{tabularx} %tabelle larghe

%------Indice----
\usepackage{hyperref} %linkare l indice
\hypersetup{hidelinks} %nascondere i link dell'indice


\title{Progetto di Programmazione a Oggetti}
\author{Sara Righetto, matricola 1174009}
\date{Anno accademico 2018/2019}

\begin{document}
    \maketitle
    \begin{abstract}
        QPlay è un applicazione che consente, tramite l'uso di un contenitore, la gestione e visualizzazione di file audio-visivi di vario tipo. L'utente può inserire, modificare, eliminare file, eseguire ricerche in base a diversi campi dati e caricare/salvare su disco la propria lista di file personale.
    \end{abstract}

    \section*{Gerarchia dei tipi}
        \includegraphics[width=1\textwidth]{gerarchia}
\paragraph{Gerarchia dei tipi}
La gerarchia è così composta: una classe base astratta denominata \textit{AudioVisual}, tre classi derivate \textit{TvSerie}, \textit{Movie}, \textit{Documentary} che implementano i metodi puri della classe base. \newline

\paragraph{AudioVisual}
I campi dati sono:
\begin{itemize}
    \item Title
    \item Plot
    \item ReleaseDate
    \item RunningTime
    \item Director
    \item Favorite
    \item Audio
    \item Video
    \item Pic %immagine di copertina
\end{itemize}
I metodi sono:
\begin{itemize}
    \item Costruttore
    \item Distruttore
    \item Clone : metodo usato da deepptr per la costruzione di copia profonda
    \item Operatore di uguaglianza e disuguaglianza
    \item isFavorite : metodo usato per sapere se un "AudioVisual" è tra i preferiti dell'utente
    \item getTotalRunningTime : metodo che restituisce il tempo totale di un AudioVisual o sotto tipo
    \item getType : restituisce il tipo dell'oggetto d'invocazione ("tipo")
    \item getQuality : Audio + Video
    \item get : genere + rating 
\end{itemize}

\paragraph{TvSerie}
I campi dati sono:
\begin{itemize}
    \item Season
    \item Episode
    \item Cast
    \item Genre 
    \item Rating 
    \item Ended 
\end{itemize}
I metodi sono:
\begin{itemize}
    \item implementa i metodi virutali puri della classe base astratta AudioVisual
\end{itemize}

\paragraph{Movie}
I campi dati sono:
\begin{itemize}
    \item Cast
    \item Genre
    \item Rating
    \item Collection 
\end{itemize}
I metodi sono:
\begin{itemize}
    \item implementa i metodi virutali puri della classe base astratta AudioVisual
    \item isCollection
\end{itemize}

\paragraph{Documentary}
I campi dati sono:
\begin{itemize}
    \item Narrator
    \item Topic : si spazia da argomento scientifico, storico a quelli biografici
\end{itemize}
I metodi sono:
\begin{itemize}
    \item implementa i metodi virutali puri della classe base astratta AudioVisual
\end{itemize}

\paragraph{User}
I campi dati sono:
\begin{itemize}
    \item Nickname
    \item TotalTime
\end{itemize}
I metodi sono:
\begin{itemize}
    \item getTotalTime : somma di tutti i RunningTime degli AudioVisual appartenenti alla lista
\end{itemize}

    \section*{Container<T>}
        Il contenitore templetizzato è stato sviluppato secondo la linea del contenitore \texttt{List} della libreria STL. \newline
        Il contenitore fa utilizzo delle classi annidate \texttt{Node} che rappresenta un nodo della lista, con la sua \textit{info}, il puntatore al nodo precedente \textit{prev} e al successivo \textit{next}, la classe \texttt{Iterator} e \texttt{Const\_Iterator} usate per scorrere gli elementi della lista.
        Come da specifica offre metodi per l'inserimento di nodi, quali \texttt{push\_front} e \texttt{push\_back}, la rimozione di nodi tramite \texttt{clear}, \texttt{erase}, \texttt{pop\_front} e \texttt{pop\_back}, e la ricerca grazie all'uso di \texttt{search}. \newline
        Sebbene il contenitore venga utilizzato istanziandolo con parametro uguale a \texttt{DeepPtr<AudioVisual>}, il parametro T può assumere qualsiasi tipo, che sia primitivo o definito dall'utente.
        %Double linked list per sostenere meglio i costi per la rimozione dei nodi in posizione arbitraria.

    \section*{XML}
        save, load, file xml

    \section*{GUI}
        Per lo sviluppo della parte relativa alla GUI è stato adottato il design pattern \textit{Model-View}. \newline
        La parte grafica si compone di una \texttt{MainWindow} principale che viene lanciata all'avvio dell'applicazione, da \texttt{AddDialog} che si occupa di inserire nuovi oggetti nella collezione, da \texttt{AudioVisualItem} che si occupa di mostrare i dettagli principali di ciascun oggetto contenuto nella lista, da \texttt{EditWidget} che si occupa di modificare un singolo elemento della lista, da \texttt{DisplayWidget} che si occupa di visualizzare tutti i dettagli di un oggetto. \newline
        %Sia per la rimozione che per la ricerca non mi è stato necessario creare altre classi e widget all'infuori di quelli sopra descritti.
        %Da qui, nella barra in alto si può scegliere se caricare, salvare od uscire dall'applicazione.

        %insert, remove, find, edit

    \section*{Ambiente di sviluppo e direttive di compilazione}
        Il progetto è stato sviluppato su sistema operativo Manjaro Linux, utlizzando l'IDE \textit{QtCreator} con il relativo framework QT v. 5.12 e compilatore GCC v. 9.1.0. \newline
        Poichè il progetto fa ampio uso di alcune \textit{keywords} di C++11 (per esempio \texttt{auto} e \texttt{nullptr}) viene fornito il file \texttt{.pro} da compilare tramite comandi \texttt{qmake} e successivamente \texttt{make} che creano il relativo file eseguibile.

    \section*{Ripartizione ore}
        La realizzazione del progetto è costata circa 50 ore ripartite nel seguente modo:
        \begin{itemize}
            \item Analisi preliminare del problema: 1h
            \item Progettazione modello e GUI: 6h
            \item Apprendimento libreria Qt: 8h
            \item Codifica modello e GUI: 31h
            \item Debugging e testing: 6h
        \end{itemize}


    

        %Obbligatorio: descrizione dei tipi e gerarchia, dell uso delle chiamate polimorfe, del formato di load e save, manuale della gui, indicazioni per compilare ed eseguire, ore effettive richieste .
\end{document}
