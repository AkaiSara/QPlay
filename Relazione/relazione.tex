\documentclass[10pt,a4paper]{article}

\usepackage[english,italian]{babel}
\usepackage[utf8]{inputenc}
\usepackage[T1]{fontenc}

\usepackage{comment}

%------Liste------
\usepackage{enumitem} %lists

%------Figure------
\usepackage{graphicx} %img/media

%------Math------
\usepackage{mathtools} %math package
\usepackage{amssymb} %pacchetto per i simboli di matematica
\usepackage{amsmath} %pacchetto di matematica
\usepackage{amsthm} %pacchetto di matematica\raggedbottom

%------CodeEnviroment----
\usepackage{listings} %coding environment 
\usepackage{xcolor} %define colors

%-------Margini------
\usepackage{geometry}
%\geometry{a4paper,top=1.5cm,bottom=2.5cm,left=1.5cm,right=1.5cm,heightrounded}
\raggedbottom

%------Tabelle------
\usepackage{booktabs} %toprule e midrule usati al posto di hline
\usepackage{float} %oggetti mobili(tabelle, figure etc)
\usepackage{longtable,tabu} %tabelle lunghe su più pagine
\usepackage{tabularx} %tabelle larghe

%------Indice----
\usepackage{hyperref} %linkare l indice
\hypersetup{hidelinks} %nascondere i link dell'indice


\begin{document}
    Obbligatorio: descrizione dei tipi e gerarchia, dell uso delle chiamate polimorfe, del formato di load e save, manuale della gui, indicazioni per compilare ed eseguire, ore effettive richieste .

    massimo 8 pagine in formato 10pt.
    \section{Introduzione}
        Descrizione dello scopo del qontainer applicato alla gerarchia

    \section{Container<T>}
        Double linked list per sostenere meglio i costi per la rimozione dei nodi in posizione arbitraria. Sviluppate le funzionalità di inserimento(push insert), rimozione(erase, pop), ricerca. Usato all' interno la "struttura" nodo, iteratore normale e costante. 

    \section{Gerarchia dei tipi}
        \documentclass[10pt,a4paper]{article}

\usepackage[english,italian]{babel}
\usepackage[utf8]{inputenc}
\usepackage[T1]{fontenc}

\usepackage{comment}

%------Liste------
\usepackage{enumitem} %lists

%------Figure------
\usepackage{graphicx} %img/media

%------Math------
\usepackage{mathtools} %math package
\usepackage{amssymb} %pacchetto per i simboli di matematica
\usepackage{amsmath} %pacchetto di matematica
\usepackage{amsthm} %pacchetto di matematica\raggedbottom

%------CodeEnviroment----
\usepackage{listings} %coding environment 
\usepackage{xcolor} %define colors

%-------Margini------
\usepackage{geometry}
%\geometry{a4paper,top=1.5cm,bottom=2.5cm,left=1.5cm,right=1.5cm,heightrounded}
\raggedbottom

%------Tabelle------
\usepackage{booktabs} %toprule e midrule usati al posto di hline
\usepackage{float} %oggetti mobili(tabelle, figure etc)
\usepackage{longtable,tabu} %tabelle lunghe su più pagine
\usepackage{tabularx} %tabelle larghe

%------Indice----
\usepackage{hyperref} %linkare l indice
\hypersetup{hidelinks} %nascondere i link dell'indice

        costituita da una classe base astratta AudioVisual e tre classi derivate Movie, Tv Serie, Documentary. è estensibile sia in orizzontale(youtube videos) che in vertiale(telenovelas). metodi virtuali puri utilizzati all'interno, quali clone utile per lo smart pointer implementato. 

    \section{GUI}
        addottato il design pattern model-view senza controller. 
        insert, remove-if, find-if, modify, save, load, file xml

    \section{Ambiente e compiler}
        file .pro (aggiungo la compilazione con c++11), os usato: debian 4.9 stretch, g++ versione 6.3.0, qt versione 5.5.1
        os: 4.19.32-1-MANJARO, g++ (GCC) 8.2.1 20181127, qt version 5.12

    \section{Ore}
        analisi preliminare del problema,progettazione
        modello e GUI, apprendimento libreria Qt, codifica modello e GUI, debugging, testing.

\end{document}